\documentclass{algo}

\usepackage{fontawesome}
\usepackage{etoolbox}
\usepackage{textcomp}
\usepackage[nodisplayskipstretch]{setspace}
\usepackage{xspace}
\usepackage{verbatim}
\usepackage{multicol}
\usepackage{soul}
\usepackage{attrib}

\usepackage{amsmath,amssymb,amsthm}

\usepackage[linesnumbered,commentsnumbered,ruled,vlined]{algorithm2e}
\newcommand\mycommfont[1]{\footnotesize\ttfamily\textcolor{blue}{#1}}
\SetCommentSty{mycommfont}
\SetKwComment{tcc}{ \# }{}
\SetKwComment{tcp}{ \# }{}

\usepackage{siunitx}

\usepackage{tikz}
\usepackage{pgfplots}
\usetikzlibrary{decorations.pathreplacing,calc,arrows.meta,shapes,graphs}

\AtBeginEnvironment{minted}{\singlespacing\fontsize{10}{10}\selectfont}
\usefonttheme{serif}

\makeatletter
\patchcmd{\beamer@sectionintoc}{\vskip1.5em}{\vskip0.5em}{}{}
\makeatother

% Math stuffs
\newcommand{\Z}{\mathbb{Z}}
\newcommand{\R}{\mathbb{R}}
\newcommand{\N}{\mathbb{N}}
\newcommand{\lcm}{\text{lcm}}
\newcommand{\Inn}{\text{Inn}}
\newcommand{\Aut}{\text{Aut}}
\newcommand{\Ker}{\text{Ker}\ }
\newcommand{\la}{\langle}
\newcommand{\ra}{\rangle}

\newcommand{\yournewcommand}[2]{Something #1, and #2}

\newenvironment{question}[1]{\par\textbf{Question #1.}\par}{}

\newcommand{\pmidg}[1]{\parbox{\widthof{#1}}{#1}}
\newcommand{\splitslide}[4]{
    \noindent
    \begin{minipage}{#1 \textwidth - #2 }
        #3
    \end{minipage}%
    \hspace{ \dimexpr #2 * 2 \relax }%
    \begin{minipage}{\textwidth - #1 \textwidth - #2 }
        #4
    \end{minipage}
}

\newcommand{\frameoutput}[1]{\frame{\colorbox{white}{#1}}}

\newcommand{\tikzmark}[1]{%
\tikz[baseline=-0.55ex,overlay,remember picture] \node[inner sep=0pt,] (#1)
{\vphantom{T}};
}

\newcommand{\braced}[3]{%
    \begin{tikzpicture}[overlay,remember picture]
        \draw [thick,decorate,decoration={brace,raise=1ex,amplitude=4pt},blue] (#2.south west-|T1.south west) -- node[anchor=west,left,xshift=-1.8ex,text=olive]{#3} (#1.north west-|T1.south west);
    \end{tikzpicture}
}

\title{Data Structures and Sorting Implementations}
\author{Sumner Evans}
\date{25 August 2023}
\institute{Colorado School of Mines}

\begin{document}

\begin{frame}{Why Do We Care?}

    Data structures allow us to hide implementation details of how our data is
    stored in memory. This is convenient for programmers because it allows us to
    interact with our data via \textit{abstractions}.
    \pause

    However, without understanding some of the implementation details, we might
    misanalyse our algorithmic complexities when using these data structures.
    \pause

    This can not only lead to you getting a bad grade on your algorithms
    homework assignments or projects, but it can also lead to you writing code
    which you \textit{think} is efficient, but is actually very slow.

\end{frame}

\begin{frame}{Lists}

    A list is a data structure which supports at least the following operations:

    \begin{itemize}
        \item \texttt{append(x)} or \texttt{push(x)}: Add the element \texttt{x}
            to the end of the list.
        \item \texttt{at(i)} or the \texttt{[i]} operator: Get the element at
            index $i$.
        \item \texttt{insert(i, x)}: Insert the element \texttt{x} at index $i$.
    \end{itemize}
    \pause

    What is the complexity of \texttt{append(x)}? \texttt{at(i)}? \\
    \texttt{insert(i, x)}?
    \pause

    \textbf{This is a trick question!} The answer depends on the list data
    structure implementation you are using.

\end{frame}

\begin{frame}{List Implementations in C++}
    In C++ there are four main list implementations:

    \begin{itemize}
        \item \href{https://en.cppreference.com/w/cpp/container/vector}{\texttt{std::vector}}
        \item \href{https://en.cppreference.com/w/cpp/container/list}{\texttt{std::list}}
        \item \href{https://en.cppreference.com/w/cpp/container/forward_list}{\texttt{std::forward\_list}}
        \item \href{https://en.cppreference.com/w/cpp/container/deque}{\texttt{std::deque}}
    \end{itemize}

    Let's look at cppreference.com to see what the documentation says about the
    complexity of the various operations.

    \pause

    \begin{block}{Question}
        With your neighbour, write down one real-world situation where you would
        use each of the four list implementations.
    \end{block}
\end{frame}

\begin{frame}{What about \{MY\_FAVOURITE\_LANGUAGE\}?}
    \begin{itemize}
        \item Python has \texttt{list} (dynamically sized array) and
            \texttt{collections.deque} (doubly-linked list).
        \item Rust has \texttt{std::vec::Vec} (dynamically sized array),
            \texttt{std::collections::VecDeque} (ring buffer), and
            \texttt{std::collections::LinkedList} (doubly-linked list).
        \item JavaScript has \texttt{Array} and \texttt{TypedArray} (both
            dynamically sized arrays).
        \item Java has \texttt{ArrayList} (dynamically sized array),
            \texttt{LinkedList} (doubly-linked list), and
            \texttt{Vector} (synchronized dynamically sized array).
        \item Go has slices (dynamically sized arrays).
    \end{itemize}
\end{frame}

\begin{frame}{Dictionaries (Maps)}

    Generally, a dictionary is a data structure which supports at least the
    following operations:

    \begin{itemize}
        \item \texttt{get(k)} or \texttt{find(k)}: Get the value associated with
            the key \texttt{k}.
        \item \texttt{set(k, v)}: Set the value associated with the key
            \texttt{k} to \texttt{v}.
        \item \texttt{remove(k)}: Remove the key \texttt{k} from the dictionary.
        \item \texttt{contains(k)}: Return whether or not the dictionary
            contains the key \texttt{k}.
    \end{itemize}

\end{frame}

\begin{frame}{Dictionary Implementations in C++}
    In C++ there are two main dictionary implementations:

    \begin{itemize}
        \item \href{https://en.cppreference.com/w/cpp/container/map}{\texttt{std::map}}
        \item \href{https://en.cppreference.com/w/cpp/container/unordered_map}{\texttt{std::unordered\_map}}
    \end{itemize}

    Let's look at cppreference.com to see what the documentation says about the
    complexity of the various operations.

    \pause

    \begin{block}{Question}
        With your neighbour, write down one real-world situation where you would
        use each of the dictionary implementations.
    \end{block}
\end{frame}

\begin{frame}{What about \{MY\_FAVOURITE\_LANGUAGE\}?}
    \begin{itemize}
        \item Python has \texttt{dict} (hash map).
        \item Rust has \texttt{std::collections::hash\_map::HashMap} (hash map),
            \texttt{std::collections::BTreeMap} (B-Tree).
        \item JavaScript has \texttt{Map} (hash map).
        \item Java has \texttt{HashMap} (hash map),
            \texttt{TreeMap} (Red-Black Tree), and
            \texttt{LinkedHashMap} (hash map with predictable iteration).
        \item Go has \texttt{map} (hash map).
    \end{itemize}
\end{frame}

\begin{frame}{Sets}
    Sets are just dictionaries without a value associated with each key (only
    the key matters).

    For each of the dictionary data structure implementations in each of the
    languages we've discussed except Go, there is a corresponding set data
    structure implementation with equivalent performance characteristics.
    \pause

    \begin{itemize}
        \item \texttt{std::set} and \texttt{std::unordered\_set} in C++.
        \item \texttt{set} in Python.
        \item \texttt{std::collections::hash\_set::HashSet} and
            \texttt{std::collections::BTreeSet} in Rust.
        \item \texttt{Set} in JavaScript.
        \item \texttt{HashSet}, \texttt{TreeSet}, and \texttt{LinkedHashSet} in
            Java.
    \end{itemize}
\end{frame}

\begin{frame}{Queues, Stacks, and Deques}
    In languages that have queues and stacks in their standard libraries, they
    are usually implemented as wrappers around a list data structure.
    \pause

    Stacks can be implemented with a list since appending and removing from the
    end of a list is $\mathcal{O}(1)$.
    \pause

    Queues are generally implemented with a deque since removal from the front
    and insertion at the back of a queue are both $\mathcal{O}(1)$ operations.
\end{frame}

\begin{frame}{Sorting}
    Most languages provide built-in sort functions. In general, you will need to
    read your programming language's documentation to find out what complexity
    guarantees the sorting functions provide.

    \begin{itemize}
        \item C++:
            \href{https://en.cppreference.com/w/cpp/algorithm/sort}{\texttt{std::sort}},
            \href{https://en.cppreference.com/w/cpp/algorithm/stable_sort}{\texttt{std::stable\_sort}},
            and
            \href{https://en.cppreference.com/w/cpp/algorithm/qsort}{\texttt{std::qsort}}.

        \item Python: \texttt{sorted} and \texttt{list.sort} use
            \href{https://en.wikipedia.org/wiki/Timsort}{Timsort}.

        \item Rust:
            \href{https://doc.rust-lang.org/std/vec/struct.Vec.html\#method.sort}{\texttt{Vec::sort}}
            and
            \href{https://doc.rust-lang.org/std/primitive.slice.html\#method.sort_unstable}{\texttt{slice::sort\_unstable}}.

        \item Go:
            \href{https://pkg.go.dev/sort\#Sort}{\texttt{sort.Sort}} and
            \href{https://pkg.go.dev/sort\#Stable}{\texttt{sort.Stable}}. Or with
            generics:
            \href{https://pkg.go.dev/slices\#Sort}{\texttt{slices.Sort}} and
            \href{https://pkg.go.dev/slices\#SortStableFunc}{\texttt{slices.SortStableFunc}}.
    \end{itemize}

    \pause

    \begin{block}{Note}
        Sorting algorithms are like crypto algorithms: you generally should not
        implement them yourself.
    \end{block}
\end{frame}

\end{document}
% Local Variables:
% TeX-command-extra-options: "-shell-escape"
% End:
